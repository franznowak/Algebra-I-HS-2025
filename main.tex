\documentclass{article}

\usepackage{graphicx}
\usepackage{amsfonts}
\usepackage{amsmath}
\usepackage{amssymb}
\usepackage{amsthm}
\usepackage{mathtools}
\usepackage{stmaryrd}
\usepackage{enumitem}
\usepackage{todonotes}
\usepackage[nameinlink]{cleveref}
\usepackage{parskip}
\usepackage{tikz-cd}

\crefname{section}{\S}{\S\S}
\crefname{table}{Tab.}{}
\crefname{figure}{Fig.}{}
\crefname{algorithm}{Alg.}{}
\crefname{equation}{Eq.}{}
\crefname{appendix}{App.}{}
\crefname{theorem}{Satz}{}
\crefname{lemma}{Lem.}{}
\crefname{corollary}{Kor.}{}
\crefformat{section}{\S#2#1#3} 
\crefname{proposition}{Prop.}{}
\crefname{definition}{Def.}{}
\crefname{conjectiure}{Con.}{}
\crefname{bemerkung}{Bem.}{}
\crefname{behauptung}{Beh.}{}

\theoremstyle{plain}
\newtheorem{definition}{Definition}
\newtheorem{lemma}{Lemma}
\newtheorem{theorem}{Satz}
\newtheorem{proposition}{Proposition}
\newtheorem{corollary}{Korollar}
\newtheorem{beispiel}{Beispiel}
\newtheorem{mytransform}{Transform}
\newtheorem{claim}{Claim}
\newtheorem{bemerkung}{Bemerkung}
\newtheorem{behauptung}{Behauptung}
\renewcommand{\proofname}{Beweis:}

\renewcommand{\ker}{\mathop{\mathrm{Kern}}}
\newcommand{\bild}{\mathop{\mathrm{Bild}}}
\newcommand{\defn}[1]{\textbf{#1}}
\newcommand{\defeq}{:=}
\newcommand{\R}{\mathbb{R}}
\newcommand{\C}{\mathbb{C}}
\newcommand{\Z}{\mathbb{Z}}
\newcommand{\norm}[1]{||#1||}
\newcommand{\vu}{\boldsymbol{u}}
\newcommand{\vv}{\boldsymbol{v}}
\newcommand{\vx}{\boldsymbol{x}}
\newcommand{\innerprod}[1]{\langle{#1}\rangle}
\newcommand{\Tr}{\mathop{\text{Tr}}}
\newcommand{\rank}{\mathop{\text{rank}}}
\newcommand{\N}{\mathcal{N}}
\newcommand{\I}{\text{I}}
\newcommand{\E}{\mathop{\mathbb{E}}}
\newcommand{\rv}[1]{\underline{\underline{#1}}}
\newcommand{\argmin}{\mathop{\text{argmin}}}
\newcommand{\argmax}{\mathop{\text{argmax}}}
\newcommand{\mlow}{\mathbf{M}^\text{low}}
\newcommand{\bigo}[1]{\mathcal{O}\left(#1\right)}
\newcommand{\noise}{\boldsymbol{w}}
\newcommand{\noiset}{\boldsymbol{\tilde{w}}}
\newcommand{\betahat}{\boldsymbol{\hat\beta}}
\newcommand{\xmax}{\tilde{x}}
\newcommand{\ug}{\leq}
\newcommand{\normal}{\trianglelefteq}
\newcommand{\sgn}{\mathrm{sign}}
\newcommand{\ord}{\mathop{\text{ord}}}
\newcommand{\teilt}{\big{|}}
\newcommand{\zykl}[1]{{<}{#1}{>}}
\newcommand{\sym}{\mathop{\text{Sym}}}
\newcommand{\id}{\mathrm{id}}
\newcommand{\ggt}{\mathop{\text{ggT}}}
\newcommand{\iso}{\simeq}
\newcommand{\isom}{\mathop\text{Isom}}
\newcommand{\trans}{\mathcal{T}}

\title{Algebra I -- {Prof. Christian Urech}}
\author{Mitschrift: Franz Nowak}
\date{Herbstsemester 2025}

\begin{document}

\maketitle

\section*{Vorlesung 1}
\begin{definition}
    Eine \defn{Gruppe} ist eine Menge $G$ zusammen mit einer Verknüpfung $*\colon G\to G, \ (g, h) \to g * h$, sodass:
    \begin{enumerate}[label=(\arabic*)]
        \item (Assoziativität) $\forall g,h,k\in G\colon (g*h)*k=g*(h*k)$
        \item (Neutrales Element) $\exists e\in G\colon g*e=e*g=g \quad \forall g\in G$
        \item (Inverses Element) $\forall g\in G \exists g^{-1}\in G$ s.d. $g*g^{-1}=g^{-1}*g=e$
    \end{enumerate}
    Eine Gruppe ist \defn{abelsch} (kommutativ), wenn $\forall g, h\in G, g*h=h*g$.
\end{definition}
Wir schreiben oft $1$ oder $1_G$ für $e$ und $gg'$ für $g*g'$ mit $g,g'\in G$. Wenn $G$ kommutativ ist, dann schreiben wir $e=0$ und $a + b$ für $a * b$. Des Weiteren sind $a^n := \overbrace{a \cdots a}^{\text{n-mal}}$ und $a^0:=1$.

\begin{bemerkung}
    Wenn $G$ assoziativ ist, dann ist $g_1g_2\cdots g_n$ eindeutig definiert (für $g_1,g_2,\ldots,g_n\in G$).
\end{bemerkung}
\begin{theorem}
    \begin{enumerate}[label=(\alph*)]
        \item Das neutrale Element ist eindeutig.
        \item Das Inverse von jedem Element ist eindeutig.
    \end{enumerate}
\end{theorem}
\begin{proof}
    \begin{enumerate}[label=(\alph*)]
        \item Seien $e, e'\in G$ neutrale Elemente. Dann ist $e=ee'=e'$.
        \item Seien $\overline{g}, g^{-1}$ Inverse von $g\in G$. Dann ist $\overline{g}=\overline{g}e=\overline{g}gg^{-1}=eg^{-1}=g^{-1}$.\looseness=-1
    \end{enumerate}
\end{proof}

\begin{theorem}
    Seien $G$ eine Gruppe und $a,b,c\in G$, sodass $ab=ac$. Dann ist $b=c$.
\end{theorem}
\begin{proof}
    $$ab=ac\implies \underbrace{a^{-1}a}_{e}b = \underbrace{a^{-1}a}_{e}c\implies b=c$$
\end{proof}

\subsubsection*{Beispiele}
\begin{itemize}
    \item Ganze Zahlen mit Addition, $(\Z, +)$ oder $\Z^+$
    \item Reelle Zahlen mit Addition, $(\R, +)$ oder $\R^+$
    \item Körper $K$ mit Addition, $(K, +)$ oder $K^+$. (Bemerkung: Keine Gruppe mit Multiplikation, wenn 0 enthalten ist.)
    \item Vektorraum $V$ mit Addition, $(V, +)$ oder $V^+$.
    \item Allgemeine lineare Gruppe, $GL_n(K)$
    \item Spezielle lineare Gruppe, $SL_n(K) \defeq \{A\in GL_n(K)\mid \det A=1\}$
    \item Orthogonale Gruppe, $O_n$
    \item Unitäre Gruppe, $U_n$
\end{itemize}

\subsubsection*{Permutationsgruppen}
Sei $\sym(M)$ die Menge der Bijektionen von einer Menge $M$ zu sich selbst, zusammen mit der Verknüpfung von Abbildungen.
Die \defn{symmetrische Gruppe} $S_n\defeq \sym(\{1,2,\ldots,n\})$ ist eine Gruppe mit $n!$ Elementen.
\begin{bemerkung}
    Jedes Element in $S_n$ ist ein Produkt von Transpositionen.
\end{bemerkung}
\paragraph{Erinnerung:} Eine \defn{Transposition} ist eine Permutation, die genau zwei Elemente vertauscht und die übrigen gleich lässt.

\begin{beispiel}
    $S_3$, die Gruppe der Permutationen von $\{1,2,3\}$.
    Seien $\sigma, \tau\in S_3$,
    $$
        \sigma\colon\begin{cases}
            1\to 2\\
            2\to 1\\
            3\to 3
        \end{cases}
        \qquad \tau\colon\begin{cases}
            1\to 2\\
            2\to 3\\
            3\to 1
        \end{cases}
    $$
    Dann sind $\sigma^2 = \id$ und $\tau^3 = \id$.
    $$\begin{rcases}
        \sigma\tau(1) = 1\\
        \tau\sigma(1) = 3
    \end{rcases} \xrightarrow{} \sigma\tau \neq \tau\sigma $$
    D.h. $S_3$ ist nicht abelsch.
\end{beispiel}

\subsection*{Untergruppen}
\begin{definition}
    Sei $G$ eine Gruppe. Eine \defn{Untergruppe} $H\leq G$ ist eine Teilmenge $H\subseteq G$ sodass 
    \begin{enumerate}[label=(\alph*)]
        \item $\forall a,b \in H, ab\in H$
        \item $1_G \in H$
        \item $\forall a\in H, a^{-1}\in H$
    \end{enumerate}
\end{definition}
\begin{bemerkung}
    Jede Untergruppe ist eine Gruppe $(H, *_H)$. $*_G$ induziert $*_H$.
\end{bemerkung}
\begin{bemerkung}
    $H \subseteq G$ mit $H\neq \{\emptyset\}$ ist eine Untergruppe von $G$ genau wenn $\forall a,b\in H,\ ab^{-1}\in H$.
\end{bemerkung}
\begin{proof}
    ``$\Rightarrow$'': klar.
    
    ``$\Leftarrow$'': Bedingung: Seien $a,b\in H$.
    \begin{enumerate}[label=(\alph*)]
            \item $\implies b^{-1}\in H\\\implies ab=a(b^{-1})^{-1}\in H$
            \item $\implies aa^{-1}\in H, d.h. 1_G\in H$
            \item $\implies 1_G a^{-1}\in H$ d.h. $a^{-1}\in H$
    \end{enumerate}    
\end{proof}

\begin{bemerkung}
    Jede Gruppe $G$ hat als Untergruppen immer $\{1\}$ (die triviale Untergruppe) und $G$ selbst. Andere Untergruppen heissen \defn{echte} Untergruppen.
\end{bemerkung}
\subsubsection*{Beispiele}
\begin{itemize}
    \item $SL_n(K) \leq GL_n(K)$
    \item $n\Z\leq \Z\quad \forall n\in\Z$
    \item Sei $S^1 \defeq \{c\in\C^*\mid |C|=1\}$. $S^1\leq \C^*$. ($\C^*\defeq (\C\backslash\{0\}, \cdot$)
    \item $B_n(K) \defeq \{A\in GL_n(K)\mid A \text{obere Dreiecksmatrix}\}$. $B_n\leq GL_n(K)$.
    \item $O_n\leq GL_n(\R)$
    \item Die alternierende Gruppe $A_n\leq S_n$ ist die Untergruppe aller Permutationen, die das Produkt einer geraden Anzahl von Transpositionen sind.
\end{itemize}
\begin{bemerkung}
    Seien $G$ eine Gruppe und $a\in G$. Dann ist $$\zykl{a}\defeq\{\ldots, a^{-2}, a^{-1}, a^0, a, a^2,\ldots\}$$ eine Untergruppe von $G$, genannt die von $a$ erzeugte \defn{zyklische Gruppe}.
\end{bemerkung}

\begin{bemerkung}
    $\zykl{a}$ ist abelsch: $a^ma^n=a^{m+n}=a^{n+m}=a^na^m$
\end{bemerkung}

\begin{lemma}
    Sei $X\subseteq\Z$ die Menge der Zahlen $n$, sodass $a^n=1$. Dann ist $X=m\Z$ für ein $m\in\Z$.
\end{lemma}
\begin{proof}
    $X$ ist eine Untergruppe von $\Z$:
    \begin{enumerate}[label=(\alph*)]
        \item Seien $m,n\in X$, dann ist $a^{m+n}=a^ma^n=1_G\implies m+n\in X$
        \item $a^0=1_G\implies 0\in X$
        \item $n\in X\implies a^{-n}=a^na^{-n}=1_G\implies -n\in X$
    \end{enumerate}
    Gemäss Übung ist $X$ von der Form $m\Z$ für ein $m\in\Z$.
\end{proof}

Falls $m\neq 0$:

Für $n\in\Z$ schreibe $n=km+r$ für ein $k\in\Z$ s.d. $0\leq r <m$. Dann ist $a^n=a^{km+r}=a^{km}a^r=a^r$.
$\implies \zykl{a}=\{1,a,\ldots,a^{m-1}\}$ und all diese Elemente sind verschieden. (Falls $a^r=a^{r'} \implies a^{r-r'}=1\implies r-r'\in m\Z\implies r=r'\quad 0\leq r,r'<m)$

Falls $m=0$:

Dann ist $\zykl{a}=\{\ldots, a^{-2},a^{-1},1, a, a^2,\ldots\}$ und alle Partitionen sind verschieden.

\section*{Vorlesung 2}

\begin{definition}
    Die \defn{Ordnung} $|G|$ einer Gruppe $G$ ist die Anzahl der Elemente in $G$ (kann $\infty$ sein).
    Die \defn{Ordnung des Elements} $a\in G$ ist $|\zykl{a}|$, wobei $\zykl{a}=\{1,a,\ldots,a^{m-1}\}$ mit $m>0$ die kleinste Zahl s.d. $a^m=1$.
\end{definition}

\subsubsection*{Beispiele}
    \begin{itemize}
        \item $A=\begin{pmatrix}
            1&1\\-1&0
            \end{pmatrix}\in GL_2(\R)$ hat Ordnung 6.
        \item $B=\begin{pmatrix}
            1&1\\1&0
        \end{pmatrix} \in GL_2(\R)$ hat Ordnung $\infty$.
    \end{itemize}

\subsection*{Homomorphismen}
\begin{definition}
    Seien $G,G'$ zwei Gruppen. Ein \defn{Homomorphismus} ist eine Abbildung $\phi\colon G\to G'$ s.d. $\phi(ab)=\phi(a)\phi(b)\quad \forall a,b\in G$.
\end{definition}
\begin{definition}
    Ein \defn{Isomorphismus} ist ein bijektiver Homomorphismus.
\end{definition}
\subsubsection*{Beispiele}
\begin{itemize}
    \item $\det\colon GL_n(K)\to K^*$
    \item signum - $\sgn\colon S_n\to\Z/2\Z,\\ \quad \sgn(x) = \begin{cases}
        0: & \text{gerade Anzahl von Transpositionen} \\
        1: & \text{ungerade Anzahl von Transpositionen}
    \end{cases}$
    \item Fixiere $a\in G$. $\phi\colon \Z\to G$, $\phi(n) = a^n$. $\phi$ ist injektiv $\Leftrightarrow$ Ord($a$) $=\infty$.
    \item $H\leq G$, die Inklusion $\iota \colon H\to G,\quad \iota(x)=x$.
\end{itemize}
\begin{theorem}
    \quad
    \begin{enumerate}[label=(\arabic*)]
        \item Falls $\phi\colon G\to G'$ und $\psi\colon G'\to G''$ Homomorphismen sind, so auch $\psi\circ\phi\colon G\to G''$.
        \item Falls $\phi\colon G\to G'$ ein Isomorphismus ist, so auch $\phi^{-1}\colon G'\to G$.
    \end{enumerate}
\end{theorem}
\begin{proof}
    \begin{enumerate}[label=(\arabic*)]
        \item $\psi\circ\phi(ab)=\psi(\phi(a)\phi(b))=\psi\circ\phi(a) \psi\circ\phi(b)$
        \item zu zeigen: $\phi^{-1}$ ist ein Homomorphismus.
        
        Seien $a',b'\in G'$. Dann gibt es $a,b\in G$ s.d. $\phi(a)=a', \phi(b)=b'$

        Es gilt $\phi(ab)=\phi(a)\phi(b)=a'b' \implies \phi^{-1}(a'b')=\phi^{-1}(a')\phi^{-1}(b')$
    \end{enumerate}
\end{proof}

\begin{bemerkung}
    Zwei zuklische Gruppen gleicher Ordnung sind immer isomorph.
\end{bemerkung}
\begin{proof}
    Seien $G=\zykl{a}, G'=\zykl{b}$ und $\phi \colon G\to G', \quad \phi(a^n)\mapsto b^n$.
    
    Falls  $|G|=|G'|$ endlich ist, so ist $G=\{1,a,\ldots,a^{m-1}\}$, $G'=\{1,b,\ldots,b^{m-1}\}$.
    Somit ist $\phi$ wohldefiniert, bijektiv und ein Homomorphismus.

    Falls $|G|=|G'|=\infty$, so ist $\phi$ wohldefiniert, bijektiv und ein Homomorphismus.\looseness=-1
\end{proof}

Wir schreiben $C_n$ für die zyklische Gruppe der Ordnung $n$.

\begin{theorem}
    Sei $\phi\colon G\to G'$ ein Homomorphismus. Dann sind $\phi(1_G)=1_{G'}$ und $\phi(a^{-1})=\phi(a)^{-1}\ \forall a\in G$
\end{theorem}
\begin{proof}
\begin{align*}
    1_G&=1_G1_G\\
    &\implies \phi(1_G)=\phi(1_G1_G)=\phi(1_G)\phi(1_G)\\
    &\underset{\text{kürzen}}{\implies}1_{G'}=\phi(1_G)
\end{align*}
Ausserdem: 
\begin{align*}
    \phi(a^{-1}\phi(a) &= \phi(a^{-1}a)=\phi(1_G)=1_{G'}\\
    &\implies \phi(a^{-1}=\phi(a)^{-1}
\end{align*}
\end{proof}

\begin{definition}
    Ein \defn{Automorphismus} ist ein Isomorphismus $\phi\colon G\to G$ von einer Gruppe $G$ zu sich selbst.
\end{definition}

\begin{beispiel}
    Für $f\in G$ definiere $\phi\colon G\to G,\quad \phi(g)\defeq fgf^{-1}$ ($fgf^{-1}$ ist das Konjugierte von $g$ unter $f$).
    $\phi$ ist ein Automorphismus.
\end{beispiel}
\begin{proof}
    Homomorphismus: $\phi(gh)=fghf^{-1}=fg(f^{-1}f)hf^{-1}=\phi(g)\phi(h)$.
    Bijektiv: $\phi^{-1}(g)=f^{-1}gf$
\end{proof}

\begin{definition}
    Für einen Homomorphismus $\phi\colon G\to G'$ definiere:

    $\bild\phi\defeq \{x\in G'\mid x=\phi(a) \text{ für ein } a\in G\}$

    $\ker\phi\defeq \{a\in G\mid \phi(a)=1\}$
\end{definition}
Übung: Zeige, dass beides Untergruppen von G' bzw. G sind.
\subsubsection*{Beispiele}
\begin{itemize}
    \item $\det\colon GL_n(K)\to K^*,\quad \ker\det=SL_n(K)$
    \item $\sgn S_N\to C_2,\quad \ker\sgn=A_n$
\end{itemize}
\begin{bemerkung}\label{bem:ker}
    Seien $\phi\colon G\to G'$ ein Homomorphismus und $a\in \ker\phi$ und $b\in G$. 
    Dann ist 
    \begin{align*}
        \phi(bab^{-1})&=\phi(b)\phi(a)\phi(b)^{-1}=1\\
        &\implies bab^{-1}\in\ker\phi
    \end{align*}
\end{bemerkung}
\begin{definition}
    Eine Untergruppe $N\leq G$ heisst \defn{Normalteiler}, falls $a\in N$ und $\forall b\in G$ $bab^{-1}\in N$.
\end{definition}
$\overset{\text{\cref{bem:ker}}}{\implies}\ker\phi$ ist immer ein Normalteiler.

\section*{Vorlesung 3}
\paragraph{Erinnerung:}
Eine Untergruppe $N\ug G$ ist ein Normalteiler, falls:
$$\forall a\in N, \forall b \in G: bab^{-1}\in N $$.
Clicker Frage zu Normalteilern $\normal$:
\begin{enumerate}
    \item $B_n(K) \ug GL_n(K)$ ist kein Normalteiler.
    \item $Z^+ \normal R^+$ ist Normalteiler (weil $R^+$ abelsch)
    \item $SL_n(K) \normal GL_n(K)$, weil $\det(ABA^{-1}) = \det(A)\det(B)\det(A)^{-1}=\det(B)$, oder bemerke, dass $SL_n(K)=\ker \det$
    \item $A_n\normal S_n$ weil $A_n=\ker\sgn$.
\end{enumerate}

\subsection*{Partitionen}

Sei $\phi\colon G\to G'$ ein Homomorphismus. Für jedes Element $h\in H$ betrachte die \defn{Faser} $\phi^{-1}(h)=\{g\in G\mid\phi(g)=h\}$ (Urbild von G in H).
Die Fasern bilden eine Partition von G.

\begin{beispiel} Sei $\phi\colon \C^*\to\R^*_{>0},\qquad \phi(z)\mapsto |z|$.\qquad
Allgemein: $\phi^{-1} = \ker \phi$.
\end{beispiel}

\begin{theorem}
    Sei $U: G\to G'$ ein Homomorphismus mit Kern $N$. Für $a,b\in G$ gilt $\phi(a)=\phi(b) \Leftrightarrow \exists n'\in N$ s.d. $b=an$, d.h. $a^{-1}b\in N$.\looseness=-1
\end{theorem}
\begin{proof}
    ``$\Rightarrow$'': Falls $\phi(a)=\phi(b)$, dann it $U(a)^{-1}\phi(b)=\phi(a^{-1}b)=1$, d.h. $\exists n\in N$, s.d. $a^{-1}b=n \implies b=an$.\looseness=-1
    
    ``$\Leftarrow$'' Falls $b=an$ für $n\in N$, dann ist $\phi(b)=\phi(a)\phi(n)=\phi(a)$.
\end{proof}
Aus dem Satz folgt, dass die Fasern von $\phi$ alle von der folgenden Form sind: $$aN=\{g\in G\mid g=an \text{ für ein } n\in N\}$$
\begin{corollary}
    Ein Homomorphismus $\phi\colon G\to G'$ ist injektiv $\Leftrightarrow \ker\phi = \{1\}$.
\end{corollary}
\begin{proof}
    ``$\Rightarrow$'' klar.

    ``$\Leftarrow$'' Man nehme an, dass der Kern $\phi= \{1\}$. $\phi(a)=\phi(b)\Leftrightarrow a^{-1}b\in\ker\phi$, d.h. $a^{-1}+b=1\implies a=b$.
\end{proof}
\subsection*{Nebenklassen}
\paragraph{Erinnerung:} Sei $X$ eine Menge. Eine \defn{Äquivalenzrelation} auf $X$ ist eine binäre  Relation $\sim$ so dass: 
\begin{enumerate}[label=\roman*)]
    \item (Transitivität) Falls $a\sim b$ und $b\sim c$, dann ist $a\sim c$.
    \item (Symmetrie) Falls $a\sim b$, so ist $b\sim a$.
    \item (Reflexivität) $a\sim a$ für alle $a\in X$.
\end{enumerate}
\paragraph{Gesehen:} Jede Äquivalenzrelation definiert eine Partition von $X$. Diese besteht aus den \defn{Äquivalenzklassen}, d.h. Teilmengen von der Form $[a]\defeq\{b\in X\mid b\sim a\}$.\looseness=-1

Sei $\overline{X}$ die Menge der Äquivalenzklassen. Dann  erhalten wir eine surjektive Abbildung $\pi\colon X\to \overline{X}, \qquad \pi(a) \defeq [a]$.
Dann ist $\pi^{-1}([a])=\{b\in X\mid b\sim a\}$.

\paragraph{Gesehen:} ``Rechnen modulo $m$''.
$\Z$ mit Äquivalenzrelation $\equiv$, wobei $a\equiv b$ falls $a-b\in m\Z$.

Menge der Äquivalenzklassen: $\Z/m\Z$.$\qquad \Z/m\Z = \{[0],[1],\ldots,[m-1]\}$.

Ausserdem können wir die Klassen in $\Z/m\Z$ miteinander addieren, so dass $[a+b]=[a]+[b]$.\looseness=-1

$\Z/m\Z$ mit Addition ist somit eine Gruppe, und die Quotientenabbildung $\pi\colon \Z\to\Z/m\Z,\quad \pi(n)\defeq[n]$ ist ein Homomorphismus.

\begin{definition}
    Sei $H\ug G$ eine Untergruppe. Eine \defn{Linksnebenklasse} von $H$ ist eine Teilmenge von der Form $aH=\{ah\mid h\in H\}$ für ein $a\in G$.
\end{definition}

\begin{beispiel}
    $m\Z^+\ug\Z^+$. Dann sind die Linksnebenklassen $m\Z$ die Teilmengen von der Form $0+m\Z, 1+m\Z,\ldots,(m-1)+m\Z$.
\end{beispiel}

Wir schreiben $a\equiv b$, falls ein $h\in H$ existiert, so dass $b=ah$, d.h. falls $b\in aH$.

\begin{theorem}
    Die Relation ``$\equiv$'' ist eine Äquivalenzrelation.
\end{theorem}
\begin{proof}
    \begin{enumerate}
        \item Falls $a\equiv b$ und $b\equiv a \implies \exists h,h'\in H$, so dass $b=ah$ und $c=bh' \implies c=a\underbrace{hh'}_{\in H}\implies c\equiv a.$
        \item falls $a\equiv b$, so $\exists h\in H$ s.d. $b=ah \implies a=b\underbrace{h^{-1}}_{\in H}\implies b\equiv a$.
        \item $a=a\cdot 1$ und $1\in H\implies a\equiv a$.
    \end{enumerate}
    $\phi\colon X\to Y$ Abbildung $\phi^{-1}(y) = \{x\in X\mid \phi(x)=y\}$ für $y\in Y$.
\end{proof}

\begin{corollary}
    Die Linksnebenklassen bilden eine Partition von G.
\end{corollary}
\begin{proof}
    $aH=bH\Leftrightarrow a\equiv b$.    
\end{proof}

\begin{definition}
    Die Anzahl der Linksnebenklassen von $H$ in $G$ ist der sogenannte \defn{Index von $H$ in $G$}. Wir schreiben $[G:H]$ für den Index. ($[G\text{ in }H]$ kann $\infty$ sein.)\looseness=-1
\end{definition}

\begin{beispiel}
    $m\geq 1,\qquad [\Z:m\Z]=m$.
\end{beispiel}

\begin{theorem}
    Sei $G$ eine endliche Gruppe und $H\ug G$. Dann ist $|G|=|H|[G:H]$.
\end{theorem}
\begin{proof}
    Die Abbildung $\phi\colon H\to aH, \qquad \phi(h)=ah$. 
    
    $\phi$ ist eine Bijektion. $\implies |H|=|aH|$.

    Die Linksnebenklassen bilden eine Partition von $G$. $\implies |G|=|H|[G:H]$
\end{proof}
Daraus folgt direkt:
\begin{corollary}[Satz von Lagrange]\label{stz:lagrange}
    Seien $G$ eine Gruppe und $H\ug G$ eine Untergruppe. Dann ist $|H|$ ein Teiler von $|G|$.
\end{corollary}
\begin{bemerkung}
    Falls $a\in G$, dann folgt mit Lagrange, dass $|\zykl{a}|\ \teilt\ |G|$, d.h. $Ord(a)$ teilt die Ordnung von $G$.
\end{bemerkung}
\begin{corollary}
    Sei $G$ eine Gruppe, s.d. $|G|$ prim ist. Sei $a\in G, a\neq 1$, dann ist $G=\zykl{a}$. 
\end{corollary}
\begin{proof}
    $\ord a \teilt p$, da $\ord a > 1$ ist, $\ord a =p$, d.h. $|\zykl{a}|=p \implies \zykl{a}=G.$
\end{proof}

\begin{corollary}\label{cor:ordnung}
    Seien $G, G'$ endliche Gruppen und $\phi\colon G\to G'$ ein Homomorphismus. Dann gilt:
    $$|G|=|\ker \phi|\cdot|\bild \phi|$$
\end{corollary}
\begin{proof}
    Gesehen: Die Linksnebenklassen von $\ker \phi$ sind die Fasern von $\phi$. 
    
    $\implies |\bild\phi|=[G:\ker\phi]$
    
    $\implies |G| = |\ker\phi|\cdot[G:\ker\phi]$
    
    \qquad $=|\ker\phi|\cdot|\bild\phi|$
\end{proof}

\begin{definition}
    Sei $G$ eine Gruppe und $H\ug G$. Die \defn{Rechtsnebenklassen} von $H$ in $G$ sind die Mengen $Ha\defeq\{ha\mid h\in H\}$.
\end{definition}

Definiere $a\equiv_R b$, falls es ein $h\in H$ gibt, so dass $b=ha$.

Dies definiert eine Äquivalenzrelation auf $G$ und die Rechtsnebenklassen sind die Äquivalenzklassen bezüglich dieser Relation. $\leadsto$  Partition von G.

\begin{theorem}
    Eine Untergruppe $H\ug G$ ist ein Normalteiler $\Leftrightarrow$ jede Linksnebenklasse ist auch eine Rechtsnebenklasse. In diesem Fall ist $aH=Ha$.
\end{theorem}
\begin{proof}
    ``$\Rightarrow$'' $H$ Normalteiler. Sei $h\in H$ und $a\in G$.
    \begin{align*}
        &\implies ah=\underbrace{(aha^{-1})}_{=:k\in H}a = ka\\
        &\implies aH\subseteq Ha
    \end{align*}
    Analog zeigt man $Ha\subseteq aH$. $\implies aH=Ha$.

    ``$\Leftarrow$'' Man nehme an, $H$ ist kein Normalteiler. 
    
    $\implies \exists h\in H, g\in G$ s.d. $aha^{-1}\notin H$, d.h. es gibt kein $h'\in H$ s.d. $ah=h'a$.
    
    $\implies ah\in aH$, aber $ah\notin Ha$, d.h. $aH\neq Ha$.

    Gleichzeitig ist $a\in aH\cap Ha\neq \emptyset$

    $\implies aH$ ist in keiner anderen Rechtsnebenklasse enthalten. D.h. Rechts- und Linksnebenklassen definieren zwei verschiedene Partitionen.
\end{proof}

\section*{Vorlesung 4}
Clicker Frage zu Homomorphismen $\phi\colon G\to G'$:
\begin{itemize}
    \item Gesehen in Übung: $\bild \phi \ug G'$.
    \item Dann folgt mit \cref{stz:lagrange}: $|\bild \phi|\teilt |G'|$ 
    \item Und mit \cref{cor:ordnung}: $|\bild \phi| \teilt |G|$.
\end{itemize}

Seien $G$ eine Gruppe und $H\ug G$ $\leadsto$ $G/H$ Linksnebenklassen von $H$ in $G$.
Können wir auf $G/H$ eine Gruppenstruktur definieren, so dass die Abbildung $\pi\colon G\to G/H, \pi(g)=gH$ ein Gruppenhomomorphismus ist?

Ja, wenn $H\normal G$ (siehe Übung).

\subsection*{Faktorgruppen}
\begin{lemma}\label{lem:sur_gruppe}
    Seien $G$ eine Gruppe und $X$ eine Menge mit einer Verknüpfung.
    Sei $\phi\colon G\to X$ eine surjektive Abbildung, so dass $\phi(ab)=\phi(a)\phi(b)\quad \forall a,b\in G$. Dann ist $X$ eine Gruppe.
\end{lemma}
\begin{proof}
    \begin{enumerate}[label=(\roman*)]
        \item Seien $u,v,w\in X$. $\exists a,b,c \in G$ s.d. $\phi(a)=u, \phi(b)=v, \phi(c)=w$.
        Dann ist 
        \begin{align*}
            u(vw)&=\phi(a)(\phi(b)\phi(c))=\phi(a)\phi(bc)\\
            &=\phi(abc)=\phi(ab)\phi(c)\\
            &=(\phi(a)\phi(b))\phi(c) = (uv)w
        \end{align*} 
        $\leadsto$ Assoziativität der Verknüpfung auf X.
        \item Sei $e:=\phi(1)$ und $u\in X$. Dann $$\exists u\in G\text{, s.d. }u=\phi(a)\implies eu=\phi(1)\phi(a)=\phi(1a)=\phi(a)=u.$$ Analog: ue=u. $\leadsto e$ ist ein neutrales Element.
        \item Sei $u\in X \implies \exists a\in G$ s.d. $u=\phi(a)$. Sei $u':=\phi(a^{-1})$. Dann ist $$u'u=\phi(a^{-1}\phi(a)=\phi(a^{-1}a)=\phi(1)=e.$$ Analog: $uu'=e$. $\leadsto$ es existieren Inverse.
    \end{enumerate}
\end{proof}
Notation: Seien $G$ eine Gruppe, $A, B \subseteq G$.
Dann definieren wir $$AB := \{ab\mid a\in A, b\in B\} \subseteq G.$$

\begin{lemma}
    Seien $G$ eine Gruppe, $N\normal G$ ein Normalteiler und $a,b \in G$. Dann ist $(aN)(bN) = abN$.
    Das Produkt von zwei Nebenklassen ist also wieder eine Nebenklasse.
\end{lemma}
\begin{proof}
    In Vorlesung 3 gesehen: $$Nb=bN\quad\forall b\in G$$
    Da $N$ eine Untergruppe ist, ist $NN=N$ (Übung).
    $$\implies (aN)(bN) = a(Nb)N=a(bN)N=abNN=abN.$$
\end{proof}
Wir erhalten also eine Verknüpfung auf die Nebenklassen.
Falls $K_1, K_2\in G/N$: 
Sei $a\in K_1, b\in K_2$. 
$\implies K_1=aN, K_2=bN$.
Dann ist $K_1K_2=abN$ (gemäss Lemma), d.h. $K_1K_2$ ist die Nebenklasse, die das Element $ab$ enthält.

\begin{theorem}
    Seien $G$ eine Gruppe und $N\normal G$.
    Mit dieser Verknüpfung bildet $G/N =:\overline G$ eine Gruppe und die Abbildung $\pi\colon G\to G/N=\overline G \quad a\mapsto aN=: \overline a$ ist ein Homomorphismus.
\end{theorem}
\begin{proof}
    Bereits beobachtet: $\pi(a)\pi(b)=(aN)(bN)=abN=\pi(ab)$.

    Aus \cref{lem:sur_gruppe} folgt, dass $\overline G=G/N$ eine Gruppe ist und daher $\pi$ ein Homomorphismus ist.
\end{proof}

\begin{corollary}
    Jeder Normalteiler $N\ug G$ ist Kern von einem Homomorphismus. Nämlich vom Homomorphismus $\pi\colon G\to G/N$.
\end{corollary}
\begin{proof}
    Das neutrale Element von $G/N$ ist $N$. $\leadsto\ker \pi=N$
\end{proof}
\begin{theorem}[erster Isomorphiesatz]\label{stz:isoeins}
    Sei $\phi\colon G\to G'$ ein surjektiver Homomorphismus und $N:=\ker\phi$. Dann ist die Gruppe $G/N$ isomorph zur Gruppe $G'$ unter dem Homomorphismus $\overline\phi\colon G/N\to G'\quad \overline a=aN\mapsto\phi(a)$
\end{theorem}
\begin{proof}
    \begin{enumerate}
        \item $\overline\phi$ ist wohldefiniert: $\phi(an)=\phi(a)\phi(n)=\phi(a)$, d.h. $\overline\phi(aN)$ hängt nicht von der Wahl des Repräsentanten ab.
        \item $\overline\phi$ ist ein Homomorphismus: 
        \begin{align*}
            \overline\phi((aN)(bN))&=\overline\phi(abN)\\
            &=\phi(ab)=\phi(a)\phi(b)\\
            &=\overline\phi(aN)\overline\phi(bN)
        \end{align*}
        \item $\overline\phi$ ist bijektiv: $\overline\phi$ ist surjektiv, da $\phi$ surjektiv ist. $\overline\phi$ ist injektiv, da $\ker\overline\phi=\{N\}$ und $N$ ist das neutrale Element in $G/N$. $\implies\overline\phi$ ist injektiv.
    \end{enumerate}
\end{proof}
\begin{definition}
    Seien $G,G'$ Gruppen, dann ist $G\times G'$ eine Gruppe mit der Verknüpfung $(a,a')(b,b')=(ab,a'b')$.
    Neutrales Element: $(1_G,1_{G'})$.
    Inverses Element: $(a,a')^{-1}=(a^{-1},a'^{-1})$.
    Es heisst das \defn{direkte Produkt} von $G$ und $G'$.\looseness=-1
\end{definition}



\section*{Vorlesung 5}

Clicker Frage: Sei $S^1 \ug \C^*$ die Untergruppe der komplexen Zahlen bestehnd aus den Elementen mit Betrag 1. Dann ist der Quotient $\C^*/S^1$ isomorph zu $\R^*_{>0}$. (Wahr)

Begründung:
Die Abbildung $\phi\colon\C^*\to R^*_{>0}, \quad z\mapsto |z|$ ist ein surjektiver Homomorphismus. 
$\ker\phi=S^1 \overset{\text{1. Isosatz}}{\implies} C^*/S^1\iso \R^*_{>0}$

Clicker Frage: Sei $G$ eine Gruppe und $H_1,H_2\ug G$ Untergruppen. Dann ist $H_1\cup H_2$ eine Untergruppe von $G$. (Wahr)

Begründung:
\begin{gather*}
    1\in H_1\cup H_2\\
    a,b\in H_1\cup H_2\implies ab\in H_1\cup H_2\\
    a^{-1}\in H_1\cup H2
\end{gather*}
AllgemeinL Falls $H_i\ug G, i\in I$ eine Familie von Untergruppen ist, so ist $\cup_{i\in I}H_i\ug G$ eine Untergruppe (selber Beweis).

\begin{definition}
    Sei $S\subseteq G$ eine Teilmenge. Dann ist $\zykl{s}\defeq \cup_{H\ug G s.d. S\subseteq H}H$ die \defn{von $S$ erzeugte Untergruppe}.
\end{definition}
\paragraph{Erinnerung:} $G,G'$ Gruppen $\leadsto$ $G\times G'$ ist Gruppe mit Verknüpfung $(a,a')(b,b')=(ab,a'b')$.
\paragraph{Bsp:} Kleinsche Vierergruppe (die "Matratzengruppe").
$$C_2\times C_2 =\{(1,1),(1,-1),(-1,1),(-1,-1)\}$$
\paragraph{Bsp:} $m,n>0$ s.d. $\ggt(m,n)=1$ dann ist $C_{mn}\iso C_m\times C_n$

Wir haben vier Homomorphismen:

% \begin{tikz}
% $G\arrow^i G\times G' \arrow^p G$
% $G'\arrow^i'       \arrow^p' G'$
% See picture.
\[
\begin{tikzcd}[row sep=large, column sep=huge]
G \arrow[dr,"i"] & & G \\
& G\times G' \arrow[ur,"p"] \arrow[dr,swap,"p'"] & \\
G' \arrow[ur,swap,"i'"] & & G'
\end{tikzcd}
\]

\begin{align*}
    i(x) &= (x,1)\\
    i'(x)&=(1,x')\\
    p(x,x')&=x\\
    p'(x,x')&=x'
\end{align*}
\begin{bemerkung}
    $i,i'$ sind injektiv, d.h. \begin{align*}
        G\times 1&=\bild i \iso G\\
        1\times G'&=\bild i'\iso G'
    \end{align*}
    $p$ und $p'$ sind surjektiv $$\ker p=1\times G', \ker p'=G\times 1$$ 
\end{bemerkung}

Sei $H$ eine Gruppe und $\phi\colon H\to G, \phi'\colon H\to G'$ Homomorphismen.
Dann ist $\Phi\colon H\to G\times G'\quad \Phi(h)=(\phi(h),\phi'(h))$ ein Homomorphismus.

Umgekehrt ist jeder Homomorphismus $\Phi\colon H\to G\times G'$ von dieser Form mit $\phi=\Phi\circ p$ und $\phi'=\Phi\circ p'$.
\begin{bemerkung}
    $\Phi(h)-(1,1) \Leftrightarrow \phi(h)=1$ und $\phi'(h)=1$ d.h. $\ker \Phi =\ker\phi\cup\ker\phi'$.
\end{bemerkung}
Seien $H,K\ug G$. Betrachte $HK=\{hk\mid h\in H, k\in K\}$. Wann ist $HK$ eine Untergruppe?
Wann ist $\pi\colon H\times K \to G \quad \pi(h,k)=hk$ ein Homomorphismus?

\begin{theorem}
    \begin{enumerate}[label=(\alph*)]
        \item Ist $H\cup K=\{1\}$, so ist $\pi$ injektiv.
        \item Ist $H$ oder $K$ ein Normalteiler, so ist $HK=KH$ und $HK$ ist eine Untergruppe von $G$.
        \item Sind $H$ und $K$ Normalteiler und gilt $H\cup K=\{1\}$ und $HK=G$ so ist $\pi\colon H\times K\to G$ ein Isomorphismus.
    \end{enumerate}
\end{theorem}
\begin{proof}
    \begin{enumerate}[label=(\alph*)]
        \item Seien $(h_1,k_1),(h_2,k_2)\in H\times K$ s.d. $h_1k_1=h_2k_2$. 
        \begin{align*}
        &\implies \underbrace{k_1k_2^{-1}}_{\in K}=\underbrace{h_1^{-1}h_2}_{\in H}\overset{H\cup K =\{1\}}{=}1\\
        &\implies k_1=k_2 \text{ und } h_1=h_2\\
        &\implies \pi \text{ ist injektiv}.
        \end{align*}
        \item oBdA. $H$ ist Normalteiler. Seien $h\in H, k\in K$.
        \begin{align*}
            &\implies kh=\underbrace{(khk^{-1})}_{\in H}k\in HK\\
            \implies KH\subset HK
        \end{align*}
        Analog: $HK\subset KH$.
        $\implies KH=HK$.
        Z.z: $HK$ ist Untergruppe.
        \begin{enumerate}[label=(\roman*)]
            \item Seien $hk, h'k'\in HK$. 
            \begin{align*}
                \implies (hk)(h'k')&=h\underbrace{(kh')}_{\in KH=HK}k'\\
                &= h(h''k'')k'\\
                &=(hh'')(k''k')\in HK
            \end{align*}
            \item $1\in HK$
            \item $hk\in HK\implies (hk)=k^{-1}h^{-1}\in kh=HK$
        \end{enumerate}
        \item Seien $h\in H, k\in K$
        \begin{align*}
        &\implies \underbrace{\underbrace{(hkh^{-1})}_{\in k}k^{-1}}_{\in K} = \underbrace{h\underbrace{(kh^{-1}k^{-1})}_{\in H}}_{\in H}\\
        &\implies hkh^-1k^-1 = 1\\
        &\implies hk=kh\\
        &\implies \pi(h_1,k_1)\pi(h_2,k_2) = h_1k_1h_2k_2=h)1h_2k_1k_2=\pi((h_1,k_1)(h_2,k_2))
        \end{align*}
        $\implies \pi$ ist Homomorphismus.
        Gemäss (a) ist $\pi$ injetiv. Da $HK=G$ ist $\pi$ surjektiv $\implies \pi$ ist Isomorphismus.
    \end{enumerate}
\end{proof}

\subsubsection*{Beispiele}
\begin{itemize}
    \item Gruppen von der Ordnung 1: nur $\{1\}$
    \item Gruppen von der Ordnung 2: nur $C_2$
    \item Gruppen von der Ordnung 3: nur $C_3$
    \item Gruppen von der Ordnung 4: $C_4, C_2\times C_2$ (s. Übung).
    \item Gruppen von der Ordnung 5: $C_5$
\end{itemize}
\begin{behauptung}
    Die einzigen Gruppen von Ordnung 6 sind $C_6$ und $S_3$ (bis auf Isomorphie).
\end{behauptung}
\begin{proof}
    Sei $G$ eine Gruppe mit $|G|=G$. Falls $G$ ein Element der Ordnung 6 enthält, so ist $G\iso C_6$.
    Ansonsten 3 mögliche Fälle:
    \begin{enumerate}[label=(\alph*)]
        \item Alle $g\in G, g\neq 1$ haben Ordnung 2
        \item Alle $g\in G, g\neq 1$ haben Ordnung 3
        \item Es gibt $g\in G$ von Ordnung 2 und $h\in G$ von Ordnung 3.
    \end{enumerate}
    Falls (a), so ist $G$ abelsch. Sei $g\in G$ 
    \begin{align*}
    &\implies \zykl{g}==\{1,g\}\normal G\\
    &\implies |G/\zykl{g}|=3\\
    &\implies G/\zykl{g}\iso C_3
    \end{align*}
    $\pi\colon G\to G/\zykl{g}$ Quotient 
    
    $\forall g\in G$ ist $\pi(g)^2=\pi(g^2)=1$. Widerspruch zu $|G/\zykl{g}|=3$.
    
    Falls (b), so gilt $g=g^{-1}$ nur wenn $g=1$..
    $\implies G=\{1,g,g^{-1}, h, h^{-1}, \ldots\}$. Nicht möglich, da $G$ eine gerade Ordnung hat.

    D.h. wir sind im Fall (c). 
    $G$ enthält $1,g,h,h^2,gh,gh^2$. (kleine Übung: Diese Elemente sind alle verschieden).
    $\implies G=\{1,g,h,h^2,gh,gh^2\}$.
    
    Wir haben $hg=gh$ oder $hg=gh^2$.
    Falls $hg=gh$, so hate $(gh)$ Ordnung 6.
    Das haben wir aber ausgeschlossen. Also ist $hg=gh^2$.
    
    Die Relation $gh=h^2g$ definiert die Verknüpfung aug G eindeutig.
    Jedes Produkt in $g$ und $h$ lässt sich mit dieser Regel in die Form $g^i h^j$ bringen, wobei $0\leq i\leq 1, 0\leq j \leq 2$.

    Im Fall (c) gibt es also höchstens eine Gruppe. Diese muss $S_3$ sein.
\end{proof}

\begin{bemerkung}
    Seien $g,h\in S_3, \quad$ mit
    $$g\colon\begin{cases}
    1\mapsto2\\
    2\mapsto3\\
    3\mapsto1
    \end{cases}\quad
    h\colon\begin{cases}
        1\mapsto2\\
        2\mapsto3\\
        3\mapsto1
    \end{cases}$$
    
    Dann ist $S_3=\{1,g,h,h^2,gh,gh^2\}$.
    \begin{bemerkung}
        Jede echte Untergruppe von $S_3$ ist zyklisch (da von Ordnung 2 oder 3).
    \end{bemerkung}
    \begin{bemerkung}
        $A_3=\zykl{h}$
    \end{bemerkung}
\end{bemerkung}

\subsection*{Symmetrie}
Isometrien von $\R^n$
\begin{definition}
    Eine \defn{Isometrie} von $\R^n$ ist eine Abbildung $f\colon\R^n\to R^n$ von der Form $f(X)=BX+a$ wobei $B\in O(n), b\in R^n$. Wir bezeichnen mit $\isom(\R^n)$ die Gruppe der Isometrien von $\R^n$.
\end{definition}
\begin{bemerkung}
    Man kann zeigen, dass Isometrien genau die Abbildungen $\R^n\to R^n$ sind, welche die Distanzen erhalten.
\end{bemerkung}
Zwei wichtige Untergruppen: 
\begin{enumerate}[label=(\arabic*)]
    \item $\trans_n\ug\isom(\R^n$: Die Untergruppe der \defn{Translationen}, d.h. Abbildung on der Form $t_a\colon X\mapsto X + a$ für $a\in R^n$. Es gilt $t_at_{a'}=t_{a+a'}$.
    \item $O \ug \isom(\R^n)$: Die Untergruppe der Isometrien von der Form $d_B\colon X\mapsto BX$ für $B\in O(n)$. Es gilt $d_Bd_{B'}=d_{BB'}$.
\end{enumerate}
Jedes $f\in \isom(\R^n)$ lässt sich eindeutig schreiben als $t_ad_B$ für $B\in O(n), a\in \R^n$.
Falls $f(X)=BX+a, g(X)=B'X+a'$, dann ist \begin{align*}
    f\circ g(X)&=B(B'X+a')+a\\
    &= BB'X + Ba'+a
\end{align*}.
d.h. falls $F=t_ad_B, g=t_{a'}+d_{B'}$, so ist 
\begin{align*}
    f\circ g&=t_ad_Bt_{a'}d_{B'}\\
    &=t_{Ba'+a}d_{BB'}.
\end{align*}
Wir haben also insbesondere Homomorphismus $\psi \colon\isom(R^n)\to O, \psi(t_ad_B)=d_B$.

$\ker\psi=\trans_n$.
\begin{bemerkung}
    Die Abbildung $\isom(R^n)\to\trans_n, t_ad_B\mapsto t_a$ ist kein Homomorphismus.
\end{bemerkung}

\end{document}
