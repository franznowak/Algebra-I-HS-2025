\documentclass{article}

\usepackage{graphicx}
\usepackage{amsfonts}
\usepackage{amsmath}
\usepackage{amssymb}
\usepackage{amsthm}
\usepackage{stmaryrd}
\usepackage{enumitem}
\usepackage{todonotes}
\usepackage{cleveref}
\usepackage{parskip}

\crefname{section}{\S}{\S\S}
\crefname{table}{Tab.}{}
\crefname{figure}{Fig.}{}
\crefname{algorithm}{Alg.}{}
\crefname{equation}{Eq.}{}
\crefname{appendix}{App.}{}
\crefname{theorem}{Thm.}{}
\crefname{lemma}{Lem.}{}
\crefformat{section}{\S#2#1#3} 
\crefname{proposition}{Prop.}{}
\crefname{definition}{Def.}{}
\crefname{conjectiure}{Con.}{}

\title{Algebra I - HS 2025}
\author{Franz Nowak}
\date{\today}

\theoremstyle{plain}
\newtheorem{definition}{Definition}
\newtheorem{lemma}{Lemma}
\newtheorem{theorem}{Satz}
\newtheorem{proposition}{Proposition}
\newtheorem{corollary}{Korollar}
\newtheorem{beispiel}{Beispiel}
\newtheorem{mytransform}{Transform}
\newtheorem{claim}{Claim}
\newtheorem{bemerkung}{Bemerkung}
\renewcommand{\proofname}{Beweis:}

\renewcommand{\ker}{\mathop{\text{Kern}}}
\newcommand{\bild}{\mathop{\text{Bild}}}
\newcommand{\defn}[1]{\textbf{#1}}
\newcommand{\defeq}{:=}
\newcommand{\R}{\mathbb{R}}
\newcommand{\C}{\mathbb{C}}
\newcommand{\Z}{\mathbb{Z}}
\newcommand{\norm}[1]{||#1||}
\newcommand{\vu}{\boldsymbol{u}}
\newcommand{\vv}{\boldsymbol{v}}
\newcommand{\vx}{\boldsymbol{x}}
\newcommand{\innerprod}[1]{\langle{#1}\rangle}
\newcommand{\Tr}{\mathop{\text{Tr}}}
\newcommand{\rank}{\mathop{\text{rank}}}
\newcommand{\N}{\mathcal{N}}
\newcommand{\I}{\text{I}}
\newcommand{\E}{\mathop{\mathbb{E}}}
\newcommand{\rv}[1]{\underline{\underline{#1}}}
\newcommand{\argmin}{\mathop{\text{argmin}}}
\newcommand{\argmax}{\mathop{\text{argmax}}}
\newcommand{\mlow}{\mathbf{M}^\text{low}}
\newcommand{\bigo}[1]{\mathcal{O}\left(#1\right)}
\newcommand{\noise}{\boldsymbol{w}}
\newcommand{\noiset}{\boldsymbol{\tilde{w}}}
\newcommand{\betahat}{\boldsymbol{\hat\beta}}
\newcommand{\xmax}{\tilde{x}}
\newcommand{\ug}{\leq}
\newcommand{\normal}{\trianglelefteq}
\newcommand{\sgn}{\mathrm{sgn}}
\newcommand{\ord}{\mathop{\text{ord}}}
\newcommand{\teilt}{\big{|}}
\newcommand{\zykl}[1]{{<}{#1}{>}}

\begin{document}

\maketitle
\section*{Vorlesung 3}
\paragraph{Erinnerung:}
Eine Untergruppe $N\ug G$ ist ein Normalteiler, falls:
$$\forall a\in N, \forall b \in G: bab^{-1}\in N $$.
Clicker Frage zu Normalteilern $\normal$:
\begin{enumerate}
    \item $B_n(K) \ug GL_n(K)$ ist kein Normalteiler.
    \item $Z^+ \normal R^+$ ist Normalteiler (weil $R^+$ abelsch)
    \item $SL_n(K) \normal GL_n(K)$, weil $\det(ABA^{-1}) = \det(A)\det(B)\det(A)^{-1}=\det(B)$, oder bemerke, dass $SL_n(K)=\ker \det$
    \item $A_n\normal S_n$ weil $A_n=\ker\sgn$.
\end{enumerate}

\subsection*{Partitionen}

Sei $\phi\colon G\to G'$ ein Homomorphismus. Für jedes Element $h\in H$ betrachte die \defn{Faser} $\phi^{-1}(h)=\{g\in G\mid\phi(g)=h\}$ (Urbild von G in H).
Die Fasern bilden eine Partition von G.

\begin{beispiel} Sei $\phi\colon \C^*\to\R^*_{>0},\qquad \phi(z)\mapsto |z|$.\qquad
Allgemein: $\phi^{-1} = \ker \phi$.
\end{beispiel}

\begin{theorem}
    Sei $U: G\to G'$ ein Homomorphismus mit Kern $N$. Für $a,b\in G$ gilt $\phi(a)=\phi(b) \Leftrightarrow \exists n'\in N$ s.d. $b=an$, d.h. $a^{-1}b\in N$.\looseness=-1
\end{theorem}
\begin{proof}
    ``$\Rightarrow$": Falls $\phi(a)=\phi(b)$, dann it $U(a)^{-1}\phi(b)=\phi(a^{-1}b)=1$, d.h. $\exists n\in N$, s.d. $a^{-1}b=n \implies b=an$.\looseness=-1
    
    ``$\Leftarrow$" Falls $b=an$ für $n\in N$, dann ist $\phi(b)=\phi(a)\phi(n)=\phi(a)$.
\end{proof}
Aus dem Satz folgt, dass die Fasern von $\phi$ alle von der folgenden Form sind: $$aN=\{g\in G\mid g=an \text{ für ein } n\in N\}$$

\pagebreak
\begin{corollary}
    Ein Homomorphismus $\phi\colon G\to G'$ ist injektiv $\Leftrightarrow \ker\phi = \{1\}$.
\end{corollary}
\begin{proof}
    ``$\Rightarrow$" klar.

    ``$\Leftarrow$" Man nehme an, dass Kern $\phi= \{1\}$. $\phi(a)=\phi(b)\Leftrightarrow a^{-1}b\in\ker\phi$, d.h. $a^{-1}+b=1\implies a=b$.
\end{proof}
\subsection*{Nebenklassen}
\paragraph{Erinnerung:} Sei $X$ eine Menge. Eine \defn{Äquivalenzrelation} auf $X$ ist eine binäre  Relation $\sim$ so dass: 
\begin{enumerate}[label=\roman*)]
    \item (Transitivität) falls $a\sim b$ und $b\sim c$, dann ist $a\sim c$.
    \item (Symmetrie) falls $a\sim b$, so ist $b\sim a$.
    \item (Reflexivität) $a\sim a$ für alle $a\in X$.
\end{enumerate}
\paragraph{Gesehen:} Jede Äquivalenzrelation definiert eine Partition von $X$. Diese besteht aus den \defn{Äquivalenzklassen}, d.h. Teilmengen von der Form $[a]\defeq\{b\in X\mid b\sim a\}$.\looseness=-1

Sei $\overline{X}$ die Menge der Äquivalenzklassen. Dann  erhalten wir eine surjektive Abbildung $\pi\colon X\to \overline{X}, \qquad \pi(a) \defeq [a]$.
Dann ist $\pi^{-1}([a])=\{b\in X\mid b\sim a\}$.

\paragraph{Gesehen:} ``Rechnen modulo $m$".
$\Z$ mit Äquivalenzrelation $\equiv$, wobei $a\equiv b$ falls $a-b\in m\Z$.

Menge der Äquivalenzklassen: $\Z/m\Z.\qquad \Z/m\Z = \{[0],[1],\ldots,[m-1]\}$.

Ausserdem können wir die Klassen in $\Z/m\Z$ miteinander addieren, so dass $[a+b]=[a]+[b]$.\looseness=-1

$\Z/m\Z$ mit Addition ist somit eine Gruppe, und die Quotientenabbildung $\pi\colon \Z\to\Z/m\Z,\quad \pi(n)\defeq[n]$ ist ein Homomorphismus.

\begin{definition}
    Sei $H\ug G$ eine Untergruppe. Eine \defn{Linksnebenklasse} von $H$ ist eine Teilmenge von der Form $aH=\{ah\mid h\in H\}$ für ein $a\in G$.
\end{definition}

\begin{beispiel}
    $m\Z^+\ug\Z^+$. Dann sind die Linksnebenklassen $m\Z$ die Teilmengen von der Form $0+m\Z, 1+m\Z,\ldots,(m-1)+m\Z$.
\end{beispiel}

Wir schreiben $a\equiv b$, falls ein $h\in H$ existiert, so dass $b=ah$, d.h. falls $b\in aH$.

\begin{theorem}
    Die Relation ``$\equiv$" ist eine Äquivalenzrelation.
\end{theorem}
\begin{proof}
    \begin{enumerate}
        \item Falls $a\equiv b$ und $b\equiv a \implies \exists h,h'\in H$, so dass $b=ah$ und $c=bh' \implies c=a\underbrace{hh'}_{\in H}\implies c\equiv a.$
        \item falls $a\equiv b$, so $\exists h\in H$ s.d. $b=ah \implies a=b\underbrace{h^{-1}}_{\in H}\implies b\equiv a$.
        \item $a=a\cdot 1$ und $1\in H\implies a\equiv a$.
    \end{enumerate}
    $\phi\colon X\to Y$ Abbildung $\phi^{-1}(y) = \{x\in X\mid \phi(x)=y\}$ für $y\in Y$.
\end{proof}

\begin{corollary}
    Die Linksnebenklassen bilden eine Partition von G.
\end{corollary}
\begin{proof}
    $aH=bH\Leftrightarrow a\equiv b$.    
\end{proof}

\begin{definition}
    Die Anzahl der Linksnebenklassen von $H$ in $G$ ist der sogenannte \defn{Index von $H$ in $G$}. Wir schreiben $[G:H]$ für den Index. ($[G\text{ in }H]$ kann $\infty$ sein).\looseness=-1
\end{definition}

\begin{beispiel}
    $m\geq 1,\qquad [\Z:m\Z]=m$.
\end{beispiel}

\begin{theorem}
    Sei $G$ eine endliche Gruppe und $H\ug G$. Dann ist $|G|=|H|[G:H]$.
\end{theorem}
\begin{proof}
    Die Abbildung $\phi\colon H\to aH, \qquad \phi(h)=ah$. 
    
    $\phi$ ist eine Bijektion. $\implies |H|=|aH|$.

    Die Linksnebenklassen bilden eine Partition von $G$. $\implies |G|=|H|[G:H]$
\end{proof}
Daraus folgt direkt:
\begin{corollary}[Satz von Lagrange]
    Sei $G$ eine Gruppe und $H\ug G$ eine Untergruppe. Dann ist $|H|$ ein Teiler von $|G|$.
\end{corollary}
\begin{bemerkung}
    Falls $a\in G$, dann folgt mit Lagrange, dass $|\zykl{a}|\ \teilt\ |G|$, d.h. $Ord(a)$ teilt die Ordnung von $G$.
\end{bemerkung}
\begin{corollary}
    Sei $G$ eine Gruppe, s.d. $|G|$ prim ist. Sei $a\in G, a\neq 1$, dann ist $G=\zykl{a}$. 
\end{corollary}
\begin{proof}
    $\ord a \teilt p$ da $\ord a > 1$ ist $\ord a =p$, d.h. $|\zykl{a}|=p \implies \zykl{a}=G.$
\end{proof}

\begin{corollary}
    $G, G'$ endliche Gruppen und $\phi\colon G\to G'$ ein Homomorphismus. Dann gilt:
    $$|G|=|\ker \phi|\cdot|\bild \phi|$$
\end{corollary}
\begin{proof}
    Gesehen: die Linksnebenklassen von $\ker \phi$ sind die Fasern von $\phi$. 
    
    $\implies |\bild\phi|=[G:\ker\phi]$
    
    $\implies |G| = |\ker\phi|\cdot[G:\ker\phi]$
    
    \qquad $=|\ker\phi|\cdot|\bild\phi|$
\end{proof}

\begin{definition}
    Sei $G$ eine Gruppe und $H\ug G$. Die \defn{Rechtsnebenklassen} von $H$ in $G$ sind die Mengen $Ha\defeq\{ha\mid h\in H\}$.
\end{definition}

Definiere $a\equiv_R b$ falls es ein $h\in H$ gibt, so dass $b=ha$.

Dies definiert eine Äquivalenzrelation auf $G$ und die Rechtsnebenklassen sind die Äquivalenzklassen bezüglich dieser Relation. $\leadsto$  Partition von G.

\begin{theorem}
    Eine Untergruppe $H\ug G$ ist ein Normalteiler $\Leftrightarrow$ jede Linksnebenklasse ist auch eine Rechtsnebenklasse. In diesem Fall ist $aH=Ha$.
\end{theorem}
\begin{proof}
    ``$\Rightarrow$" $H$ Normalteiler. Sei $h\in H$ und $a\in G$.
    \begin{align*}
        &\implies ah=\underbrace{(aha^{-1})}_{=:k\in H}a = ka\\
        &\implies aH\subset Ha
    \end{align*}
    Analog zeigt man $Ha\subset aH$. $\implies aH=Ha$.

    ``$\Leftarrow$" Man nehme an, $H$ ist kein Normalteiler. 
    
    $\implies \exists h\in H, g\in G$ s.d. $aha^{-1}\notin H$, d.h. es gibt kein $h'\in H$ s.d. $ah=h'a$.
    
    $\implies ah\in aH$, aber $ah\notin Ha$, d.h. $aH\neq Ha$.

    Gleichzeitig ist $a\in aH\cap Ha\neq \emptyset$

    $\implies aH$ ist in keiner anderen Rechtsnebenklasse enthalten. D.h. Rechts- und Linksnebenklassen definieren zwei verschiedene Partitionen.
\end{proof}

\end{document}
